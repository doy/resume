% vim:foldmethod=marker commentstring=%%%s
% license {{{
% This work is licensed under the Creative Commons
% Attribution-NonCommercial-ShareAlike License. To view a copy of this license,
% visit http://creativecommons.org/licenses/by-nc-sa/1.0/ or send a letter to
% Creative Commons, 559 Nathan Abbott Way, Stanford, California 94305, USA.
% This file is adapted from Todd Courtesan's resume, at
% http://www.courtesan.com/todd/resume.html
% }}}
% preamble {{{
\documentclass[letterpaper]{article}
\usepackage{jesse_resume}
\hypersetup{hidelinks}
% }}}
\begin{document}
% Header {{{
\resheader{Jesse Luehrs}
          {http://tozt.net/}
          {doy@tozt.net\hspace{0.5in}}
          {(618) 616-6287}
          {70 America St. \#1R}
          {Providence, RI 02903}
% }}}
% Education {{{
\resheading{Education}
\begin{itemize}
    % UIUC {{{
    \item \ressubheading{University of Illinois at Urbana-Champaign, College of Engineering}{Urbana, IL}
                        {Bachelor of Science in Computer Science with Minor in Mathematics}{August 2004--May 2008}
        \begin{minipage}[t]{\textwidth/2-0.2in}
            \begin{itemize}
                \resitem{Overall GPA: 3.61, Technical GPA: 3.81}\vspace{-7pt}
                \resitem{James Scholar in Engineering (2004--2005)}\vspace{4pt}
            \end{itemize}
        \end{minipage}
        \begin{minipage}[t]{\textwidth/2-0.2in}
            \begin{itemize}
                \resitem{Dean's List (Fall 2004--Fall 2006)}\vspace{-7pt}
                \resitem{Graduated with Honors}\vspace{4pt}
            \end{itemize}
        \end{minipage}
    % }}}
\end{itemize}
% }}}
% Work Experience {{{
\resheading{Work Experience}
\begin{itemize}
    % Infinity Interactive {{{
    \item \ressubheading{Infinity Interactive (\url{http://iinteractive.com/})}{Manhasset, NY (telecommuting)}
                        {Senior Programmer}{February 2010--present} \vspace{6pt} \linebreak
        I am in charge of a large, legacy codebase which handles employee
        engagement survey registration and reporting, and I have also written
        and deployed many smaller sites myself, mostly using Perl. Since we
        rely heavily on open source software, a large portion of my time is
        also devoted to maintaining various open source projects, as well as
        developing new open source software that could be useful in the future.
        For instance, I contributed to Moose, Plack, and Perl through work, and
        developed OX, Text::Handlebars, and p5-mop for use in work. These
        projects are described more fully below.
    % }}}
    % UIUC Hydrogeology Lab {{{
    \item \ressubheading{UIUC Hydrogeology Lab (\url{http://www.gwb.com/})}{Urbana, IL}
                        {Visiting Research Programmer}{February 2006--February 2010} \vspace{6pt} \linebreak
        I worked on the Geochemists' Workbench, a geochemistry software suite
        written in C++ and Tcl/Tk. I added support for several new image output
        formats, including PDF, SVG, and Adobe Illustrator, as well as
        improving the existing PostScript support by adding the ability to
        embed TrueType fonts. I also helped add parallel processing support to
        several scientific calculations, using OpenMP. I designed a new
        XML-based configuration file format for our applications and ported our
        calculation applications to Linux, to allow them to be run on large
        clusters. Finally, I implemented a testing framework for our
        calculation applications using Perl's Test::More.
    % }}}
\end{itemize}
% }}}
% Projects {{{
\resheading{Projects}

All of my personal open source work is available on GitHub
(\url{https://github.com/doy}). My Perl open source work is also available on
the CPAN (\url{https://metacpan.org/author/DOY}).

\begin{itemize}
    % Text::Handlebars {{{
    \item \resshortsubheading{Text::Handlebars (\url{https://github.com/doy/text-handlebars})}{2013--present} \vspace{6pt} \linebreak
        I am the author of Text::Handlebars, a port of the Handlebars.js
        templating language to Perl. It uses a custom parser on top of the
        Xslate template engine framework. It supports nearly the entire feature
        set of the JavaScript implementation, and we used it at Infinity
        Interactive to ease the transition of one of our large web applications
        from client side templates to server side templates.
    % }}}
    % Reply {{{
    \item \resshortsubheading{Reply (\url{https://github.com/doy/reply})}{2013--present} \vspace{6pt} \linebreak
        I am the author of Reply, a customizable and lightweight REPL for Perl.
        It provides features like pluggable tab completion, automatic class
        loading and refreshing, history support, and (through the Carp::Reply
        module) automatically launching a REPL when an exception is thrown. It
        can be easily extended through a powerful plugin system.
    % }}}
    % Plack {{{
    \item \resshortsubheading{Plack (\url{http://plackperl.org/})}{2012--present} \vspace{6pt} \linebreak
        I am a member of the core development team for Plack, the reference
        implementation of the PSGI specification for Perl web
        server/application interaction (similar to Python's WSGI and Ruby's
        Rack). I have contributed to the design of PSGI, as well as
        implementing my own PSGI-based web framework called OX (below).
    % }}}
    % p5-mop {{{
    \item \resshortsubheading{p5-mop (\url{https://github.com/stevan/p5-mop-redux})}{2011--present} \vspace{6pt} \linebreak
        I am one of the lead developers for p5-mop, which is a prototype for a
        new object system for the Perl core. It provides features such as
        attribute declarations, method signatures, and roles to bring Perl up
        to the level of other modern languages, but it also includes a fully
        featured meta object protocol, for an added level of power and
        customizability.
    % }}}
    % Perl {{{
    \item \resshortsubheading{Perl (\url{http://www.perl.org/})}{2011--present} \vspace{6pt} \linebreak
        I was the release manager for the 5.17.1 development release of Perl,
        and I have also contributed many bug fixes to the Perl core. I am
        currently working on preparing the p5-mop project (above) for
        integration into the Perl core.
    % }}}
\end{itemize}
\pagebreak
\resheading{Projects (continued)}
\begin{itemize}
    % OX {{{
    \item \resshortsubheading{OX (\url{https://github.com/iinteractive/OX})}{2011--present} \vspace{6pt} \linebreak
        I am the lead author of OX, a web framework for Perl based on the PSGI
        specification, which uses the Bread::Board dependency injection system
        to manage application components. We have used it internally at
        Infinity Interactive for many client projects. In addition to writing
        most of the framework itself, I also wrote a series of advent calendar
        posts documenting it, which can be seen at
        \url{http://ox.iinteractive.com/advent/}.
    % }}}
    % Moose {{{
    \item \resshortsubheading{Moose (\url{http://moose.perl.org/})}{2009--present} \vspace{6pt} \linebreak
        I am a member of the lead development team for Moose, a module which
        provides advanced object orientation capabilities for Perl. I was also
        the release manager from 2011--2012. I wrote several extensions for
        Moose, including MooseX::NonMoose, which allows classes built with
        Moose to easily interoperate with other types of classes, and
        MooseX::Aliases, which allows Moose attributes to be referred to by
        different names.
    % }}}
    % TAEB {{{
    \item \resshortsubheading{TAEB (\url{http://taeb.github.io/})}{2008--present} \vspace{6pt} \linebreak
        I am one of the lead framework developers for TAEB, a Perl framework
        for programmatic interaction with NetHack. I am also the primary
        developer for the leading AI written for TAEB. I developed several
        standalone Perl modules, including Graph::Implicit, which implements
        several useful graph algorithms, and IO::Pty::Easy, which provides a
        simple read/write interface for interacting with pseudo terminals.
    % }}}
    % Dungeon Crawl Stone Soup {{{
    \item \resshortsubheading{Dungeon Crawl Stone Soup (\url{http://crawl.develz.org/})}{2009--2012} \vspace{6pt} \linebreak
        I am a member of the development team for Dungeon Crawl Stone Soup, a
        roguelike game written in C++ and Lua. I contributed several features
        throughout the game, and I was also the release manager for the 0.6
        release.
    % }}}
    % Volition {{{
    \item \resshortsubheading{System for Defining, Documenting and Recording Game Events (\url{http://volition-inc.com/})}{2007--2008} \vspace{6pt} \linebreak
        This is a library written in C which can be added to games in order to
        track arbitrary events and report them to a remote server, for use in
        gameplay testing. This project was completed for Volition as my senior
        project, and was used as part of their testing process for Saints Row
        2.
    % }}}
\end{itemize}
% }}}
% Talks {{{
\resheading{Talks}

Slides and videos (where available) for these talks can be found at \url{http://tozt.net/talks.html}.

\begin{itemize}
    \item \resshortsubheading{Dependency Injection with Bread::Board (50 min)}{YAPC::NA 2012, YAPC::EU 2012} \vspace{6pt}\linebreak
        This talk provides an overview of dependency injection, and gives
        concrete examples of it using the Bread::Board and
        Bread::Board::Declare Perl modules.
    \item \resshortsubheading{OX - the hardest working two letters in Perl (50 min)}{YAPC::NA 2011} \vspace{6pt}\linebreak
        This talk describes the OX web framework for Perl (mentioned above in
        the Projects section), including a conceptual overview and usage
        examples.
    \item \resshortsubheading{Extending Moose (50 min)}{YAPC::NA 2010} \vspace{6pt}\linebreak
        This talk goes into detail describing Moose's meta object protocol,
        including what it is, how it works, and how you can extend it.
\end{itemize}
% }}}
% Skills {{{
\resheading{Skills}
\begin{description}
    \item[Languages:] I am fluent in C, C++, Perl, Lua, and shell, and I am proficient in JavaScript, HTML/CSS, Scala, Rust, Tcl/Tk, and LaTeX.\vspace{-6pt}
\item[Tools:] Make, vim, git, Firefox
\end{description}
% }}}
\end{document}
