% vim:foldmethod=marker commentstring=%%%s
% license {{{
% This work is licensed under the Creative Commons
% Attribution-NonCommercial-ShareAlike License. To view a copy of this license,
% visit http://creativecommons.org/licenses/by-nc-sa/1.0/ or send a letter to
% Creative Commons, 559 Nathan Abbott Way, Stanford, California 94305, USA.
% This file is adapted from Todd Courtesan's resume, at
% http://www.courtesan.com/todd/resume.html
% }}}
% preamble {{{
\documentclass[letterpaper]{article}
\usepackage{jesse_resume}
\hypersetup{hidelinks}
% }}}
\begin{document}
% Header {{{
\resheader{Jesse Luehrs}
          {https://tozt.net/}
          {doy@tozt.net\hspace{0.5in}}
          {(618) 616-6287}
          {142 E. 16th St. \#19D}
          {New York, NY, 10003}
% }}}
% Work Experience {{{
\resheading{Work Experience}
\begin{itemize}
    % Stripe {{{
    \item \ressubheading{Stripe (\url{https://stripe.com/})}{Remote}
                        {Staff Engineer}{January 2015--June 2021}
        \begin{itemize}
            \item Moved our CI infrastructure from a proprietary system to
                Jenkins. This included rewriting the testing infrastructure of
                our main monolith to run tests in parallel with process
                isolation to allow for better reproducibility and the ability
                to scale out automatically to many worker machines. This was
                able to reduce turnaround time for tests by over 80\%, and
                provided a path to scaling out to more machines to keep test
                times down.
            \item Converted all of our internal infrastructure to use an
                installation of Confidant
                (\url{https://lyft.github.io/confidant/}) for secrets storage
                and distribution, giving us much more control over which
                people and machines had access to our secrets.
            \item Implemented an authentication service in go which allowed
                users to sign arbitrary data as their own identity in a way
                that machines could independently verify. This allowed us to
                remove almost all use of GPG at Stripe, which eliminated a
                large class of tooling issues related to deployments.
            \item Contributed to importing all of our low level infrastructure
                which had originally been set up via custom tooling (or by
                hand) into Terraform, allowing us to (mostly) automate
                creation of new AWS accounts.
            \item Contributed to our rollout of Envoy for service-to-service
                communication, giving us automatic, transparent mutual TLS for
                almost all internal traffic. Additionally, used features
                provided by Envoy to implement a blue/green deploy mechanism
                which greatly improved speed and reliability of deploys for
                our critical services.
            \item Implemented a fleetwide service in go for running
                maintenance commands on servers (running puppet, restarting
                services, etc), which reduced the time needed for running
                these types of commands from several days in some cases to
                under 5 minutes. Additionally, designed a secure protocol for
                these types of actions which ensured that the end services
                would not perform any actions without first ensuring that the
                request was logged in a separate secure append-only logging
                system.
        \end{itemize}
    % }}}
    % Infinity Interactive {{{
    \item \ressubheading{Infinity Interactive (\url{https://iinteractive.com/})}{Remote}
                        {Senior Programmer}{February 2010--August 2014}
        \begin{itemize}
            \item Maintained a large, legacy codebase which handled employee
                engagement survey registration and reporting for several large
                companies.
            \item Wrote and deployed many small websites for clients, using
                Perl and Javascript.
            \item Developed and maintained various open source projects used in
                our client work, including Moose and Plack (see below).
        \end{itemize}
    % }}}
    % UIUC Hydrogeology Lab {{{
    \item \ressubheading{UIUC Hydrogeology Lab (\url{https://www.gwb.com/})}{Urbana, IL}
                        {Visiting Research Programmer}{February 2006--February 2010}
        \begin{itemize}
            \item Worked on the Geochemists' Workbench, a geochemistry software
                suite written in C++ and Tcl/Tk.
            \item Added support for several new image output formats including
                SVG and PDF, and added features to existing ones, including
                adding font embedding support to our PostScript output.
            \item Contributed to adding parallel processing support to several
                scientific calculations, using OpenMP.
            \item Ported our calculation applications from Windows to Linux, to
                allow them to be run on large supercomputing clusters.
            \item Implemented a testing framework for our calculation
                applications, using Perl's Test::More.
        \end{itemize}
    % }}}
\end{itemize}
% }}}
% Open Source {{{
\resheading{Open Source}

\small{A more complete list of my projects can be found on my website
(\url{https://tozt.net/projects.html} and \url{https://git.tozt.net/}). All of
my personal open source work is also available on GitHub
(\url{https://github.com/doy}), and you can also find my Rust open source work
on crates.io (\url{https://crates.io/users/doy}) and my Perl open source work
on MetaCPAN (\url{https://metacpan.org/author/DOY}).}\normalsize

\begin{itemize}
    % nbsh {{{
    \item \resshortsubheading{nbsh (\url{https://github.com/doy/nbsh})}{2021--present} \linebreak \linebreak
        I am currently developing an advanced new shell using Rust and Tokio
        which integrates aspects of terminal multiplexers to provide a user
        experience more similar to notebooks (such as Jupyter) than traditional
        shells.
    % }}}
    % rbw {{{
    \item \resshortsubheading{rbw (\url{https://github.com/doy/rbw})}{2020--present} \linebreak \linebreak
        I wrote and maintain rbw, an unofficial command-line client for the
        Bitwarden password manager. rbw is written in Rust, and uses a
        background agent (in a similar style to ssh-agent or gpg-agent) to keep
        credentials persistently in memory.
    % }}}
    % Reply {{{
    \item \resshortsubheading{Reply (\url{https://github.com/doy/reply})}{2013--2016} \linebreak \linebreak
        I wrote Reply, a lightweight and extensible REPL for Perl. It includes
        many useful features such as tab completion and history support.
    % }}}
    % Dungeon Crawl Stone Soup {{{
    \item \resshortsubheading{Dungeon Crawl Stone Soup (\url{https://crawl.develz.org/})}{2009--2016} \linebreak \linebreak
        I was a member of the development team for Dungeon Crawl Stone Soup, a
        roguelike game written in C++ and Lua. I contributed several features
        to the game and was also the release manager for the 0.6 release.
    % }}}
    % Plack {{{
    \item \resshortsubheading{Plack (\url{https://plackperl.org/})}{2012--2013} \linebreak \linebreak
        I was a member of the core development team for PSGI and Plack, the
        specification for Perl web server/application interaction (similar to
        Python's WSGI and Ruby's Rack).
    % }}}
    % Perl {{{
    \item \resshortsubheading{Perl (\url{https://www.perl.org/})}{2011--2013} \linebreak \linebreak
        I was the release manager for the 5.17.1 development release of Perl,
        and I have also contributed many bug fixes. I was also a lead developer
        on the p5-mop project, a prototype of a new object system for Perl.
    % }}}
    % Moose {{{
    \item \resshortsubheading{Moose (\url{https://metacpan.org/dist/Moose})}{2009--2013} \linebreak \linebreak
        I was a member of the development team for Moose, which provides
        advanced object orientation capabilities to Perl. I was also the
        release manager from 2011--2012.
    % }}}
    % TAEB {{{
    \item \resshortsubheading{TAEB (\url{https://taeb.github.io/})}{2008--2011} \linebreak \linebreak
        I was one of the lead framework developers for TAEB, a Perl framework
        for programmatic interaction with the game NetHack. I was also the
        primary developer for the leading bot written with TAEB framework.
    % }}}
\end{itemize}
% }}}
% Education {{{
\resheading{Education}
\begin{itemize}
    % Recurse Center {{{
    \item \ressubheading{Recurse Center (\url{https://recurse.com/})}{New York, NY}
                        {Student}{September 2014--November 2014}
    % }}}
    % UIUC {{{
    \item \ressubheading{University of Illinois at Urbana-Champaign, College of Engineering}{Urbana, IL}
                        {Bachelor of Science in Computer Science with Minor in Mathematics}{August 2004--May 2008}
        \begin{minipage}[t]{\textwidth/2-0.2in}
            \begin{itemize}
                \resitem{Overall GPA: 3.61, Technical GPA: 3.81}\vspace{-7pt}
                \resitem{James Scholar in Engineering (2004--2005)}\vspace{4pt}
            \end{itemize}
        \end{minipage}
        \begin{minipage}[t]{\textwidth/2-0.2in}
            \begin{itemize}
                \resitem{Dean's List (Fall 2004--Fall 2006)}\vspace{-7pt}
                \resitem{Graduated with Honors}\vspace{4pt}
            \end{itemize}
        \end{minipage}
    % }}}
\end{itemize}
% }}}
% Talks {{{
\resheading{Talks}

\small{Slides and videos (where available) for these talks can be found at
\url{http://tozt.net/talks.html}.}\normalsize

\begin{itemize}
    \item \resshortsubheading{Introduction to Rust (50 min)}{YAPC::NA 2014} \linebreak \linebreak
        This talk describes the Rust programming language, touching on
        its major features and design philosophies that make it
        interesting.
    \item \resshortsubheading{Dependency Injection with Bread::Board (50 min)}{YAPC::NA 2012, YAPC::EU 2012} \linebreak \linebreak
        This talk provides an overview of dependency injection, and
        gives concrete examples of it using the Bread::Board module for Perl.
    \item \resshortsubheading{OX - the hardest working two letters in Perl (50 min)}{YAPC::NA 2011} \linebreak \linebreak
        This talk describes the OX web framework for Perl, including a
        conceptual overview and usage examples.
    \item \resshortsubheading{Extending Moose (50 min)}{YAPC::NA 2010} \linebreak \linebreak
        This talk goes into detail describing Moose's meta object
        protocol, including what it is, how it works, and how you can extend
        it.
\end{itemize}
% }}}
% Skills {{{
\resheading{Skills}
\begin{description}
    \item[Languages:] I am fluent in Rust, Ruby, Go, C, Perl, Lua, and shell.
    \item[Tools:] git, vim, make, Jenkins, Terraform, Puppet, Docker
\end{description}
% }}}
\end{document}
