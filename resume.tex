% vim:foldmethod=marker commentstring=%%%s
% license {{{
% This work is licensed under the Creative Commons
% Attribution-NonCommercial-ShareAlike License. To view a copy of this license,
% visit http://creativecommons.org/licenses/by-nc-sa/1.0/ or send a letter to
% Creative Commons, 559 Nathan Abbott Way, Stanford, California 94305, USA.
% This file is adapted from Todd Courtesan's resume, at
% http://www.courtesan.com/todd/resume.html
% }}}
% preamble {{{
\documentclass[letterpaper,10pt]{article}
\usepackage{jesse_resume}
\addtolength{\oddsidemargin}{-0.375in}
\addtolength{\textwidth}{0.75in}
\addtolength{\topmargin}{-.2in}
\addtolength{\textheight}{0.4in}
% }}}
\begin{document}
% Heading {{{
\resheader{Jesse P Luehrs}
          {http://tozt.net/}
          {doy@tozt.net\hspace{0.5in}}
          {(618) 616-6287}
          {702 W. Green St., Apt \#2}
          {Urbana, IL 61801}
% }}}
% Education {{{
\resheading{Education}
\begin{itemize}
% UIUC {{{
\item \ressubheading{University of Illinois at Urbana-Champaign, College of Engineering}
                    {Urbana, IL}
                    {Bachelor of Science in Computer Science}
                    {Aug. 2004--May 2008}
    \begin{minipage}[t]{\textwidth/2-0.2in}
    \begin{itemize}
    \resitem{Overall GPA: 3.61, Technical GPA: 3.81}\vspace{-7pt}
    \resitem{James Scholar in Engineering (2004--2005)}\vspace{4pt}
    \end{itemize}
    \end{minipage}
    \begin{minipage}[t]{\textwidth/2-0.2in}
    \begin{itemize}
    \resitem{Dean's List (Fall 2004--Fall 2006)}\vspace{-7pt}
    \resitem{Graduated with Honors}\vspace{4pt}
    \end{itemize}
    \end{minipage} % }}}
\end{itemize} % }}}
% Work Experience {{{
\resheading{Work Experience}
\begin{itemize}
% UIUC Hydrogeology Lab {{{
\item \ressubheading{UIUC Hydrogeology Lab}
                    {Urbana, IL}
                    {Visiting Research Programmer}
                    {February 2006--present}
    \begin{itemize}
    \resitem{Worked on the Geochemists' Workbench, a geochemistry software suite written in C++ and Tcl/Tk.}
    \resitem{Added or enhanced support for several different image output formats including PDF, SVG, and PostScript, including adding TrueType font embedding to PDF and PostScript files.}
    \resitem{Helped add parallel processing support to several scientific calculations, using OpenMP.}
    \resitem{Designed a new XML-based configuration file format for our applications.}
    \resitem{Ported our calculation applications to Linux, to allow them to be run on large clusters.}
    \resitem{Implemented a unit testing framework for our calculation applications in Perl, using Test::More.}
    \end{itemize} % }}}
% Smile-A-While {{{
\item \ressubheading{Smile-A-While Amusements}
                    {Traveling, Illinois/Indiana/Missouri}
                    {Concessions manager}
                    {Summer 2004 and 2005}
    \begin{itemize}
    \resitem{Managed several amusement games on the Luehrs' Ideal Rides carnival.}
    \end{itemize} % }}}
\end{itemize}
% }}}
% Projects {{{
\resheading{Projects}
\begin{itemize}
% Moose {{{
\item \resshortsubheading{Moose (http://moose.perl.org/)}{2009--Present}
    \begin{itemize}
    \resitem{Member of the Moose Cabal, the lead development team for Moose.}
    \resitem{Wrote several extensions for Moose, including MooseX::NonMoose, which allows classes built with Moose to easily interoperate with other types of classes, and MooseX::Aliases, which allows Moose attributes to be referred to by different names.}
    \end{itemize} % }}}
% Bot::Games {{{
\item \resshortsubheading{Bot::Games (http://github.com/doy/bot-games)}{2009--Present}
    \begin{itemize}
    \resitem{Bot::Games is an IRC bot framework written in Perl, designed for multiplayer game moderation.}
    \resitem{Uses Moose extensively to provide a clean and extensible plugin system for adding games.}
    \end{itemize} % }}}
% TAEB {{{
\item \resshortsubheading{TAEB (http://taeb.sartak.org/)}{2008--Present}
    \begin{itemize}
    \resitem{TAEB is a Perl framework (using Moose) for programmatic interaction with NetHack (http://nethack.org/).}
    \resitem{Primary developer for the leading AI written for TAEB.}
    \resitem{Developed several standalone Perl modules in the course of development, including Graph::Implicit, which implements several useful graph algorithms, and IO::Pty::Easy, which provides a simple read/write interface for interacting with pseudo-terminals.}
    \end{itemize} % }}}
% Smithy {{{
\item \resshortsubheading{Smithy (http://sourceforge.net/projects/smithy/)}{2008}
    \begin{itemize}
    \resitem{Smithy is a cross-platform map editor for the Aleph One engine (http://marathon.sourceforge.net/), written in OCaml.}
    \resitem{Contributed several GUI improvements, including writing custom widgets using LablGTK.}
    \end{itemize} % }}}
% %hm-command {{{
%%\item \resshortsubheading{hm-command (http://tozt.net/code/hm-command/)}{2008--Present}
    %%\begin{itemize}
    %%\resitem{hm-command is a script for interacting with Hiveminder (http://hiveminder.com).}
    %%\resitem{It is written in Perl, and uses the Net::Hiveminder module to interact with Jifty's REST interface.}
    %%\end{itemize} % }}}
% Volition {{{
\item \resshortsubheading{System for Defining, Documenting and Recording Game Events (http://volition-inc.com/)}{2007--2008}
    \begin{itemize}
    \resitem{This is a library written in C which can be added to games in order to track arbitrary events and report them to a remote server, for use in gameplay testing.}
    \end{itemize} % }}}
% %LuaSignal {{{
%%\item \resshortsubheading{LuaSignal (http://luaforge.net/projects/lua-signal/)}{2007--2008}
    %%\begin{itemize}
    %%\resitem{This is a library for Lua written mostly in C, which provides POSIX signal handling support to Lua scripts.}
    %%\end{itemize} % }}}
% LuaIRC {{{
\item \resshortsubheading{LuaIRC (http://luaforge.net/projects/luairc/)}{2006--2008}
    \begin{itemize}
    \resitem{LuaIRC is a fully-featured IRC framework written in Lua.}
    \resitem{Supports all standard IRC functionality, including CTCP and DCC.}
    \end{itemize} % }}}
% %Illinexus {{{
%%\item \resshortsubheading{Illinexus (http://www.illinexus.org/)}{2006--Present}
    %%\begin{itemize}
    %%\resitem{I am the system administrator for Illinexus, which provides a free web hosting service for UIUC students and registered student organizations.}
    %%\end{itemize} % }}}
% %Interhack {{{
%%\item \resshortsubheading{Interhack (http://interhack.us/)}{2007}
    %%\begin{itemize}
    %%\resitem{Interhack is a program written in object-oriented Perl, using the Calf object system (a stripped down version of Moose). It provides user interface enhancements for the game NetHack.}
    %%\resitem{For Interhack I also developed the Perl module IO::Pty::Easy, which allows for spawning subprocesses on a pseudo-terminal, and provides access to stdin/stdout of that subprocess through a simple read/write interface.}
    %%\end{itemize} % }}}
% %Sonnet {{{
%%\item \resshortsubheading{Sonnet (http://tozt.net/code/sonnet/)}{2006}
    %%\begin{itemize}
    %%\resitem{Sonnet is an implementation of a Forth variant in Lua.}
    %%\end{itemize} % }}}
% %LSLua {{{
%%\item \resshortsubheading{LSLua}{2005--2006}
    %%\begin{itemize}
    %%\resitem{LSLua is a module for the alternative Windows shell LiteStep, which provides Lua scripting capabilities to theme authors. It is written in C++, using the Lua C API.}
    %%\end{itemize} % }}}
% %Ars Physica {{{
%%\item \resshortsubheading{Ars Physica (https://www-s.acm.uiuc.edu/wiki/space/Ars+Physica)}{2004--2005}
    %%\begin{itemize}
    %%\resitem{Ars Physica is a 3d puzzle game written in C++. I added Lua bindings to it, and used them to extend certain aspects of the game.}
    %%\end{itemize} % }}}
\end{itemize}
% }}}
% Skills {{{
\resheading{Skills}
\begin{description}
\item[Languages:]
\begin{minipage}[t]{6.5in}
\begin{description}
\item Proficient in C, C++, Perl (CPAN id: DOY), Lua, Bash, OCaml
\item Working knowledge of Ruby, Tcl/Tk, JavaScript, LaTeX, HTML/CSS, PostScript, sed
\end{description}
\end{minipage}
\item[Operating Systems:] Linux (Arch, Debian, Gentoo), Windows (2000, XP)
% TODO: other appropriate tools? should i mention vimscripting?
\item[Tools:] Make, Vim, Microsoft Visual Studio, Cygwin, Darcs, Subversion, Git
\end{description}
% }}}
% Activities {{{
\resheading{Activities}
\begin{description}
% TODO: need to find some better stuff for here...
\item[Inline Insomniacs:] I was the webmaster for the Inline Insomniacs rollerblading club from 2005 until 2007.
\item[Falling Illini:] I was a member of the Falling Illini skydiving club from 2007 until 2008.
\end{description}
% }}}
\end{document}
